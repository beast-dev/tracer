\documentclass{bioinfo}
\copyrightyear{2012}
\pubyear{2012}

% amsmath package, useful for mathematical formulas
\usepackage{amsmath}
% amssymb package, useful for mathematical symbols
\usepackage{amssymb}

\usepackage{graphicx}
\usepackage{subfigure} 
% cite package, to clean up citations in the main text. Do not remove.
\usepackage{cite}

\usepackage{url}

\begin{document}
\firstpage{1}

% Title must be 150 characters or less
\title[Tracer 1.6]{Bayesian MCMC diagnostics and summarization using Tracer 1.6}

\author[Rambaut \textit{et~al}]{ Andrew Rambaut\,$^{1}$, Alexei J.~Drummond\,$^{2,3}$, Dong Xie\,$^{2,3}$, Marc A.~Suchard\,$^{4,5,6}$}

\address{
$^{1}$Institute of Evolutionary Biology, University of Edinburgh, Edinburgh, UK\\
$^{2}$Department of Computer Science, University of Auckland, Auckland, NZ\\
$^{3}$Centre for Computational Evolution, University of Auckland, Auckland, NZ\\
$^{4,5}$Departments of Biomathematics and Human Genetics, David Geffen School of Medicine at UCLA, and \\
$^{6}$Department of Biostatistics, UCLA Fielding School of Public Health, University of California, Los Angeles, USA \\
}

\history{Received on XXXXX; revised on XXXXX; accepted on XXXXX}

\editor{Associate Editor: XXXXXXX}

\maketitle


% Please keep the abstract between 250 and 300 words
\begin{abstract}

\section{Motivation:}
Computational evolutionary biology, statistical phylogenetics and coalescent-based population genetics are becoming increasingly central to the analysis and understanding of molecular sequence data. 
\section{Results:}
%We describe here a software package, Tracer version 1.6, which ... 
We therefore develop a software package Tracer (version 1.6) for analysing the trace files generated by Bayesian MCMC runs (that is, the continuous parameter values sampled from the chain). It can be used to analyse runs of BEAST \citep{drummond2007beast,drummond2012bayesian}, BEAST2 \citep{bouckaert2014beast2}, MrBayes \citep{ronquist2012mrbayes}, RevBayes \citep{hohna2016revbayes}, LAMARC \citep{kuhner2006lamarc}, Migrate \citep{beerli2006comparison} and possibly other MCMC programs.

\section{Availability:}
Tracer is open-source under the GNU lesser general public license and available at 
\url{https://github.com/beast-dev/tracer} 
and  \href{http://tree.ed.ac.uk/software/tracer}{\url{http://tree.ed.ac.uk}}.

\section{Contact:} 
\href{a.rambaut@ed.ac.uk}{\url{a.rambaut@ed.ac.uk}}, 
\href{alexei@cs.auckland.ac.nz}{\url{alexei@cs.auckland.ac.nz}}
and 
\href{msuchard@ucla.edu}{\url{msuchard@ucla.edu}}

\end{abstract}

\section*{Introduction}

Blah blah 

\begin{itemize}
\item Importance of posterior summaries (as scientifically relevant marginalization of the full posterior distribution)
\item Full posterior distribution is high-dimensional and (often) difficult to visualize
\item Posterior simulator agnostic -- highly used with \textsc{MrBayes} as well
\end{itemize}

\section*{Design and Implementation}

The design of Tracer is illustrated in Figure~\ref{fig:tracer}. It can analyse either a single MCMC log or combine the shared traces from multiple logs. 
The list of traces in each log are created from the sampling parameters logged during a MCMC process, which include their statistical summaries from their values at all states logged. 

\begin{figure}[ht]
\includegraphics[width=.38\textwidth]{./figures/tracer.pdf}  
\caption{The design of Tracer}
\label{fig:tracer}
\end{figure}

Tracer can handle 3 trace types, also known as variable types \citep{mendenhall2012introduction}: real, integer, and categorical.
The real trace is used to handle quantitative continuous variables, which is the most common type in logged parameters.
The integer is dealing with quantitative discrete variables, and the categorical trace is for qualitative categorical variables. 
% check about defination 
These three types can be automatically assigned when the log file is imported. But they can also be changed later. 

Multiple traces can be selected as shown in Figure~\ref{fig:multitrace}, which will overlay the plots for the different traces allowing comparisons to be made. Similarly multiple trace files can be selected as well to compare different runs. If multiple trace files have the same trace names then a ``Combined'' trace will automatically appear, which can be selected as well as the individual trace files.

\begin{figure}[ht]
\includegraphics[width=.5\textwidth]{./figures/multitrace.png}  
\caption{A comparison of multiple traces}
\label{fig:multitrace}
\end{figure}

There are four analysis tabs to choose from either a selected trace (sampling parameter) or multiple traces:

\begin{figure}[ht]
\subfigure[Estimates]{\includegraphics[width = .223\textwidth, height = 3cm]{./figures/frequency.pdf}}
\subfigure[Density]{\includegraphics[width = .23\textwidth, height = 3cm]{./figures/multiKDE.pdf}}\\
\subfigure[Joint-Marginal]{\includegraphics[width = .23\textwidth, height = 3cm]{./figures/joint-marginal.pdf}}
\subfigure[Trace]{\includegraphics[width = .23\textwidth, height = 3cm]{./figures/trace.pdf}}
\caption{Image exported from Tracer's four analysis tabs}
\label{fig:4tabs}
\end{figure}

\begin{itemize}
\item Estimates - this shows the mean, standard deviation, confidence intervals, effective sample size, and other statistics about the selected trace(s). A frequency distribution will also be plotted for a single selected trace, such as Figure~\ref{fig:4tabs}.a. Interval bars for quantitative traces (real or integer) or violin plots for categorical traces will be drawn for a comparison, such as Figure~\ref{fig:multitrace}, if multiple traces are selected.

\item Density - this shows the Bayesian posterior density plot for the selected trace(s), if they are continuous, the kernel density estimates will be available, for example, in Figure~\ref{fig:4tabs}.b.

\item Joint-Marginal - this only appears if exactly 2 traces are chosen. It then plots one against the other to look at their joint-marginal distribution, for example, in Figure~\ref{fig:4tabs}.c.

% trace is duplicate to trace, use trajectory?
\item Trace - this shows the trajectory of the selected trace(s) against state or generation number as it can be seen in Figure~\ref{fig:4tabs}.d. It can be used to check mixing, choose a suitable burn-in and look for trends that might suggest problems with convergence.

\end{itemize}

In addition to illustrating joint-marginal distribution between two continuous traces, we newly invented a TangHuLu chart to visualise the joint probability between two integer or categorical traces using coloured bubbles, as shown in Figure~\ref{fig:tanghulu}.a. The size of circle is proportional to the joint probability, the blue coloured circle is in the credible set of given a probability threshold default to 0.95, the red is in the set not in the credible set. The tile background is coloured if there is any circle, in case the very small circle wouldn't be missed out. The coloured background can also help to show the area of the credible set and non-credible set.

Furthermore, box-and-whisker plots are used to display the joint-marginal distribution between one continuous trace and one integer or categorical trace as shown in Figure~\ref{fig:tanghulu}.b. 

\begin{figure}[ht]
\subfigure[TangHuLu chart]{\includegraphics[width = .23\textwidth, height = 3cm]{./figures/tanghulu.pdf}}
\subfigure[boxplot]{\includegraphics[width = .23\textwidth, height = 3cm]{./figures/boxplot.pdf}}
\caption{Visualisation of joint-marginal distribution (a) between two integer or categorical traces using TangHuLu chart, and (b) between one continuous trace and one integer or categorical trace using box-and-whisker plot.}
\label{fig:tanghulu}
\end{figure}


Tracer also provides demographic reconstruction resulting in a graphical plot to estimate the distribution of effective population sizes over time, given a MCMC log from a Bayesian framework or additionally with the tree log depending on the model.
For example, ``Demographic Reconstruction" performs reconstructions of the distribution for a number of models, such as constant size, exponential growth \citep{drummond2002estimating}, and logistic growth. %citation?
Besides ``Bayesian Skyline Reconstruction'' reconstructs a generalised Bayesian skyline plot (e.g. Figure~\ref{fig:flu}.a) describing demographic history under canonical scenarios \citep{drummond2005bayesian},
``GMRF Skyride Reconstruction'' and ``SkyGrid Reconstruction'' are respectively available for a temporal smoothing method using the Gaussian Markov random field (GMRF) model  \citep{minin2008smooth} and a generalisation of the GMRF model \citep{gill2012improving}.
``Lineages Through Time'' plots quantiles of the number of lineages against time, which describes the change of diversification of all organisms over time. %citation?
All these models are available in BEAST \citep{drummond2007beast}. 

\begin{figure}[ht]
%\includegraphics[width=.5\textwidth]{./figures/fluBSP.pdf}  
\subfigure[Bayesian Skyline]{\includegraphics[width = .22\textwidth]{./figures/fluBSP.pdf}}
\subfigure[Lineages Through Time]{\includegraphics[width = .22\textwidth]{./figures/fluLTT.pdf}}
\caption{The plot (a) performs Bayesian Skyline reconstruction to estimate population sizes over time and (b) Lineages Through Time , from ``Flu'' data, where time starts from year 1997 at the 0 position and the unit is year.}
\label{fig:flu}
\end{figure}

Tracer recently offers a fairly general solution of looking at conditional posterior distributions. This, specifically, supports for BSSVS forms of model averaging, in which some parameters are only in the likelihood when their sub-model is ``indicated" by some indicator function that is usually a discrete, integer or boolean variable. In that case the posterior of the parameter should not include the states when it was sampled only in the prior because the sub-model it belonged to was ``turned off". This is relevant for EBSP, Random Local Clocks model, microsatellite model averaging and relaxed clock model averaging methods all from my group in last couple of years. It is also relevant for BSSVS in phylogeography as well depending on how the state is logged.

% TODO more ?

\section*{Example}

Even users of \textsc{MrBayes} and \textsc{RevBayes} find Tracer handy.  

\section*{Availability and Future Directions}

We make the Tracer package available in both executable and source code forms.  Tracer requires Java version 1.6 or greater and executables for Windows, Mac OS and Linux platforms are located at \url{http://beast.bio.ed.ac.uk} which serves as the main page for the package. This page also links to a sizable list of self-contained, step-by-step tutorials covering basic to advance usage of Tracer to summarize posterior distributions of a large set of phylogenetic models simulated using BEAST.  For example, popular tutorials describe how to use Tracer to generate marginal parameter summaries and infer population dynamics trajectories over time.

Github houses the Tracer's version-controlled source code within \url{https://github.com/beast-dev/tracer} and links to two GoogleGroup discussion groups related to Tracer.  
The first is the ``beast-users" group (\url{http://groups.google.com/group/beast-users}) with over 1,500 members. 
%At the time of writing, forty-seven developers belong to the ``beast-dev" group that facilitates BEAST development across three continents.

Future development directions for Tracer focus on \ldots



\section*{Old Text}

%You can also select the "Demographic Analysis" from the Analysis menu - This plots the distribution of demographic population sizes over time for a number of models (constant size, exponential growth \& logistic growth) that are available in BEAST. This involves you selecting the traces for each parameter of the model. You should only select the model that was actually run under BEAST (e.g., if you ran an exponential growth model, you shouldn't plot the constant population size model).

%The "Analysis" menu also contains options for performing Bayesian Skyline reconstructions and for calculating Bayes Factors between runs.

Feature list:
\begin{itemize}
\item KDEs
\item cheap marginal likelihood estimators
\item demographic trajectory reconstruction
\end{itemize}

%Model selection HME and AICM have been removed 
New features:
\begin{itemize}
\item Three trace types: real, integer, string
\item Kernel density estimates
\item Conditional posterior distribution
%\item HME
%\item AICM
\end{itemize}

\begin{figure}[H]
\includegraphics[width=.5\textwidth]{./figures/comp-95HPD.pdf}  
\caption{The comparison of  95\% HPD intervals from multi-trace}
\label{fig:comp95HPD}
\end{figure}



\begin{figure}[H]
\includegraphics[width=.5\textwidth]{./figures/jointPrInt.png}  
\caption{The Joint probability table of two integer traces}
\label{fig:int:jointpr}
\end{figure}

%\begin{figure}[ht]
%\includegraphics[width=.5\textwidth]{./figures/frequency.pdf}  
%\caption{A frequency distribution for a continuous trace}
%\label{fig:freq}
%\end{figure}
%\begin{figure}[ht]
%\includegraphics[width=.5\textwidth]{./figures/multiKDE.pdf}  
%\caption{KDE of multiple traces}
%\label{fig:multiKDE}
%\end{figure}
%\begin{figure}[ht]
%\includegraphics[width=.5\textwidth]{./figures/joint-marginal.pdf}  
%\caption{The joint-marginal distribution of two selected traces}
%\label{fig:trace}
%\end{figure}
%\begin{figure}[ht]
%\includegraphics[width=.5\textwidth]{./figures/trace.pdf}  
%\caption{The selected traces against state}
%\label{fig:trace}
%\end{figure}


\subsubsection*{Diagnostics:} Blah blah 

\subsubsection*{Demographic reconstruction:} 

%\subsubsection*{Model selection:}

\subsubsection*{Conditional posterior distribution:}

%Fairly general solution to looking at conditional posterior distributions;

%Specifically, support for BSSVS forms of model averaging, in which some parameters are only in the likelihood when their submodel is "indicated" by some indicator function that is usually a discrete, integer or boolean variable. In that case the posterior of the parameter should not include the states when it was sampled only in the prior because the submodel it belonged to was "turned off". This is relevant for EBSP, Random Local Clocks model, microsatellite model averaging and relaxed clock model averaging methods all from my group in last couple of years. Also relevant for BSSVS in phylogeography as well depending on how the state is logged.
%\section*{Taken from the Tracer website}

%Tracer is a program for analysing the trace files generated by Bayesian MCMC runs (that is, the continuous parameter values sampled from the chain). It can be used to analyse runs of BEAST \citep{drummond2007beast,drummond2012bayesian}, BEAST2 \citep{bouckaert2014beast2}, MrBayes \citep{ronquist2012mrbayes}, RevBayes \citep{hohna2016revbayes}, LAMARC \citep{kuhner2006lamarc}, Migrate \citep{beerli2006comparison} and possibly other MCMC programs.

%Although Tracer can be used with programs other than BEAST, users may find it useful to join the BEAST users mailing list. This is used to announce new versions and advise users about bugs and problems.

%You can join the mailing list here:
%http://groups.google.com/group/beast-users

%The website for BEAST (and Tracer) is here:
%http://beast.bio.ed.ac.uk/

%At present there is no detailed manual for this application, you will simply have to play around and see what happens. Basically you can select the trace file in the top left of the window, the individual parameter in the bottom left and the analysis appears on the right.

%There are 4 analysis tabs to choose from:

%Estimates - this shows the mean, stdev, confidence intervals and other statistics about the selected parameter. A frequency distribution will also be plotted.
%Density - this shows the Bayesian posterior density plot for the selected parameter.
%Joint-Marginal - this only appears if exactly 2 parameters are chosen (hold down shift to select multiple parameters). It then plots one against the other to look at their joint-marginal distribution.
%Trace - this shows the trace of the parameter against state or generation number. Use this to check mixing, choose a suitable burn-in and look for trends that might suggest problems with convergence.
%Multiple parameters can be selected by holding down the shift key. This will overlay the plots for the different parameters allowing comparisons to be made. You can also select multiple trace files as well to compare different runs. If multiple trace files have the same trace names then a "Combined" trace will automatically appear. This can be selected as well as the individual trace files.

%You can also select the "Demographic Analysis" from the Analysis menu - This plots the distribution of demographic population sizes over time for a number of models (constant size, exponential growth \& logistic growth) that are available in BEAST. This involves you selecting the traces for each parameter of the model. You should only select the model that was actually run under BEAST (e.g., if you ran an exponential growth model, you shouldn't plot the constant population size model).

%The "Analysis" menu also contains options for performing Bayesian Skyline reconstructions and for calculating Bayes Factors between runs.

%The "Print" function in the "File" menu will print the current graph or table and the "Export Data" function can be used to export the data from the plots for use in another graphic package.

%To export the currently displayed graphic use the "Export PDF" function in the "File" menu.


\section*{Acknowledgments}

We thank the National Evolutionary Synthesis Center (NESCent) for sponsoring a working group (Software for Bayesian Evolutionary Analysis) that facilitated the development of Tracer version 1.6. 
This work was supported in part by funding from the Marsden Trust, NSF (DMS 0856099), NIH (R01 GM086887, R01 HG006139), The Royal Society of London, BBSRC (BB/H011285/1) and the Wellcome Trust (WT092807MA).

%\section*{References}
% The bibtex filename
\bibliographystyle{natbib}
\bibliography{tracer16}

\end{document}

